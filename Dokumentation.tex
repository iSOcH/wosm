\documentclass[11pt,a4paper,parskip=half]{scrartcl}
\usepackage{ngerman}
\usepackage[utf8]{inputenc}
\usepackage[colorlinks=false,pdfborder={0 0 0}]{hyperref}
\usepackage{graphicx}
\usepackage{caption}
\usepackage{longtable}
\usepackage{float}
\usepackage{textcomp}
\usepackage{geometry}
\usepackage{amsmath}
\usepackage{amssymb}
\geometry{a4paper, left=30mm, right=25mm, top=30mm, bottom=35mm} 
\usepackage{listings}
\lstset{breaklines=true, breakatwhitespace=true, basicstyle=\scriptsize, numbers=left}
\title{Workshop System Management}
\author{Tobias Lerch, Yanick Eberle, Pascal Schwarz}
\begin{document}
\maketitle
\newpage

\section{Netzwerk}
\subsection{Netzwerkdiagramm}
\subsection{IP Dual-Stack Konzept}
\subsubsection{IPv4}
Wir unterscheiden zwischen drei verschiedenen Netzwerke. Das interne Netzwerk, das DMZ Netzwerk und das öffentliche Netzwerk. Wir verwenden für die DMZ und das interne Netzwerk verschiedene Netzwerkklassen um die Netze schnell unterscheiden zu können. Folgende IP-Adressierung und Maskierung werden wir verwenden.

\textbf{Internes Netzwerk}\\
!!!!Create Latex Table!!!!\\
VLAN 10 Server: 10.0.10.0 255.255.255.0 Gateway 10.0.10.1\\
VLAN 20 Administratoren: 10.0.20.0 255.255.255.0 Gateway 10.0.20.1\\
VLAN 30 Entwicklung: 10.0.30.0 255.255.255.0 Gatetway 10.0.30.1\\
VLAN 40 Verkauf: 10.0.40.0 255.255.255.0 Gatetway 10.0.40.1\\
VPN Clients: 10.0.99.0 255.255.255.0\\
Infrastructure LAN: 10.100.0.0 255.255.255.252

\textbf{DMZ}\\
DMZ LAN: 172.16.0.0 255.255.255.0 Gateway 172.16.0.1

\textbf{Internet}\\
WAN: 209.165.50.0 255.255.255.0 Gateway 209.165.50.1

\subsubsection{IPv6}
Da die Hosts über das Internet direkt erreichbar sein sollen, werden wir globale IPv6 Adressen mit dem Site Prefix /64 verwenden

\textbf{Internes Netzwerk}\\
!!!!Create Latex Table!!!!\\
VLAN 10 Server: 2005:2013:FF:A10::/64 Gateway2005:2013:FF:A10::1\\
VLAN 20 Administratoren: 2005:2013:FF:A20::/64 Gateway 2005:2013:FF:A20::1\\
VLAN 30 Entwicklung: 2005:2013:FF:A30::/64 Gateway 2005:2013:FF:A30::1\\
VLAN 40 Verkauf: 2005:2013:FF:A40::/64 Gateway 2005:2013:FF:A40::1\\
Infrastructure LAN: 2005:2013:FF:A0::/64

\textbf{DMZ}\\
!!!!Create Latex Table!!!!\\
DMZ LAN: 2005:2013:FF:B0::/64 Gateway 2005:2013:FF:B0::1/64

\textbf{Internet}\\
!!!!Create Latex Table!!!!\\
WAN: 2005:209:165:50::/64 Gateway 2005:209:165:50::1/64

\subsection{Routing}
\subsubsection{Core Router}
Der Core Router hat nur default-routen konfiguriert. Sämtlicher Datenverkehr wird an die Firewall gesendet.\\
!!!!Create Latex Table!!!!\\
IPv4: 0.0.0.0 0.0.0.0 next Hop 10.100.0.2\\
IPv6: ::/0 next Hop 2005:2013:FF:A0::2

\subsubsection{Firewall}
Die default Route auf der Firewall würde normalerweise auf den Router des Service Providers weitergeleitet. Da wir in der Simulation aber keinen solchen haben, werden keine default Routen konfiguriert. Die Firewall sendet somit nur den Verkehr für das interne Netzwerk an den Core Router.\\
!!!!Create Latex Table!!!!\\
IPv4: 10.0.0.0 255.255.0.0 next Hop 10.100.0.1 (Die einzelnen VLANs wurden hier zu einem /16 Netz zusammengefasst)\\
IPv6: 2005:2013:FF:A10::/64 next Hop 2005:2013:FF:A0::1\\
2005:2013:FF:A20::/64 next Hop 2005:2013:FF:A0::1\\
2005:2013:FF:A30::/64 next Hop 2005:2013:FF:A0::1\\
2005:2013:FF:A40::/64 next Hop 2005:2013:FF:A0::1

\subsection{NAT}
Network Address Translation wird für IPv4 verwendet um den internen Clients Zugriff ins Internet zu gewähren und um den Webserver in der DMZ vom Internet aus zugänglich zu machen. Für den Internetzugriff der Clients wird ein Port Address Translation (PAT) konfiguriert, damit nur eine Public IP-Adresse verwendet werden muss. Für den Webserver wird ein statisches NAT mit einer zusätzlichen Public IP-Adresse konfiguriert.\\

Webserver: statisches NAT interne IP: 172.16.0.11 - öffentliche IP: 209.165.50.2\\
Interne Hosts: dynamisches NAT overload: interner Range: 10.0.0.0 255.255.0.0 - öffentliche IP 209.165.50.1 (Outside IF IP der Firewall)

\subsection{VTP}
\subsection{Spanning-Tree}
\subsection{VPN IPsec Remote Access}
\subsection{Serverkonzept}
!!!!Create Latex Table!!!!\\
Name					OS					IP			Gateway		Services\\
WOSMGR1LANSRV			Windows Server 2008 R2		10.0.10.21		10.0.10.1		AD / DNS / DHCP / Fileserver\\
WOSMGR1LANAdmin		Windows 7				10.0.20.21		10.0.20.1\\
WOSMGR1LANEntwicklung		Windows 7				10.0.30.21		10.0.30.1\\
WOSMGR1LANVerkauf		Windows 7				10.0.40.21		10.0.40.1\\
WOSMGR1DMZSRV			Windows Server 2008 R2		172.16.0.21		172.16.0.1		HTTP / HTTPS / FTP\\
WOSMGR1INETSRV			Windows Server 2008 R2		209.165.50.21	209.165.50.1	HTTP / HTTPS / FTP\\
WOSMGR1INETPC			Windows 7				209.165.50.22	209.165.50.1\\

\section{Sicherheit}
\subsection{Konzept}
Um die Sicherheit unseres Netzes zu gewähtleisten, haben wir uns entschieden verschiedene Sicherheitsstufen zu definieren. Dabei verfolgen wir eine High Security Strategie. Die höchste Sicherheitsstufe 'Stufe 1' gilt für die normalen User. Die zweite Sicherheitsstufe 'Stufe 2' gilt für die Server. Die dritte Sicherheitsstufe 'Stufe 3' gilt für die Administratoren.\\
Bei der Sicherheitsstufe Stufe 1 wird nur das nötigste zugelassen und alles andere blockiert. Die User dürfen über Ports 80 und 443 im Internet surfen, sowie FTP Verbindungen über Port 21 und 22 öffnen. Zudem werden eingehende DHCP Antworten über den Port UDP 68 zugelassen.\\
Bei der Sicherheitsstufe Stufe 2 wird alles zugelassen, was die Server benötigen. Dabei wird aus den VLANs 20, 30 und 40 alles zugelassen. Aus der DMZ wird nur der Port 389 für LDAP zugelassen.\\
Bei der Sicherheitsstufe Stufe 3 wird zusätzlich zu den in Stufe 1 zugelassenen Ports noch der Port 22 im internen Netz und in die DMZ zur Verwaltung der Netzwerkgeräte zugelassen. Zudem ist ist Internet für die Administratoren alles offen.\\
Die definierten Sicherheitsstufen wurden mithilfe verschiedener ACL's umgesetzt. Die definierten Regeln (Auflistung oben nicht abschliessend) der ACL's sind im folgenden Kapitel ersichtlich.\\
\\
Die ACL's werden möglichst nahe an der Quelle angewendet. Somit sind alle ACL's welche den Zugriff der verschiedenen internen VLAN's in irgend ein anderes Netz regeln auf dem Core Switch auf den Interfaces in Richtung 'in' angewendet. Alle ACL's die den Zugriff in die DMZ, rsp. von der DMZ in ein anderes Netz regeln auf der ASA angewendet. Alle ACL's die den eingehenden Traffic aus dem Internet regeln sind ebenfalls auf der ASA angewendet.\\
\\
Mit einer Stateful Firewall ist ein höherer Konfigurationsaufwand verbunden, aber gleichzeitig auch eine höhere Sicherheit. Da wir eine High Security Strategie verfolgen, ist die Stateful Variante besser geeignet für unsere Zwecke.\\
\subsection{Firewall}
ACL's!!

\section{Bedrohungsmodel}
\subsection{TCP DoS (Syn-Flooding)}
\subsubsection{Bedrohung}
Beim TCP 3-Way Handshake wird zuerst eine Anfrage an einen Server gesendet, indem ein TCP Paket mit dem Flag SYN verschickt wird. Der Server als Empfänger dieses TCP SYN Pakets verarbeitet dieses und sendet ein TCP Paket mit den Falgs SYN und ACK zurück. Er merkt sich dabei in einer SYN-Liste, mit wem er ein 3-Way Handshake begonnen hat. Wenn derInitiator der Verbindung das TCP Paket mit den Flags SYN und ACK empfängt, verarbeitet er dieses und sendet zur Bestätigung ein Paket mit dem Flag ACK. Sobald der Server das Packet mit dem Flag ACK erhalten hat, wird der Eintrag in der SYN-Liste gelöscht.\\
Ein Angreifer sendet 100 SYN-Anfragen pro Sekunde an einen bestimmten Server. Dabei setzt er eine andere Source IP Adresse, sodass die Antwort nicht zum Angreifer kommt. Da sich der Server merkt, mit wem er einen 3-Way Handshake begonnen, diese aber nicht abschliessen kann, da nie eine Bestätigung mit dem Flag ACK eintrifft, wird der Arbeitsspeicher des Server gefüllt. Sobald der Speicher gefüllt ist, kann dieser keine weiteren Verbindungen mehr aufnehmen oder stürtzt ab.
\subsubsection{Gegenmassnahme}
Um einen Webserver vor diesem Angriff zu schützen, kann die Anzahl der Verbindungen in der Warteschlange vergrössert werden. Zudem kann das Timeout reduziert werden um Speicher freizugeben.

\subsection{ICMP ‘smurf attack’: Denial of Service}
\subsubsection{Bedrohung}
Ein Angreifer sendet ein ICMP Packet mit einer Echo-Anfrage an eine oder mehrere Broadcasts und verwendet als Absenderadresse die IP Adresse des Servers (Opfer). Die Broadcastanfrage wird an alle Hosts in betroffenen Netz weitergeleitet. Die Hosts senden daraufhin ein die Echo-Antwort an den Server (Opfer). Der Server empfängt nun so viele Echo Antworten dass der Server nicht mehr reagiert und abstürtzt.
\subsubsection{Gegenmassnahme}
Um diese Attacke abzuwehren, kann ICMP blockiert werden. So ist sichergestellt, dass keine Echo Antworten den Server erreichen.

\subsection{Viren / Würmer / Trojaner}
\subsubsection{Bedrohung}
Programme, welche vertrauliche Infomrationen stehlen, Schaden auf den Hosts anrichten oder die Kontrolle über einen Host übernehmen und ihn für eigene Zwecke einsetzen. Zudem können diese Programme zum Beispiel als SMTP Relay fungieren und SPAM Nachrichten versenden, wodurch die Public IP auf einer Blackliste gelistet werden kann.
\subsubsection{Gegenmassnahme}
Um sich gegen Viren, Würmer und Trojaner zu schützen, muss ein Anti-Virenprogramm auf jedem Host installiert werden.

\subsection{DNS Cache poisoning}
\subsubsection{Bedrohung}
Ein Angreifer bringt bei einem DNS Server gefälschte Daten in den Cash. Wenn nun ein Benutzer auf diese Daten zugreift, wird dieser auf manipulierte Seiten weitergeleitet. Der Angreifer kan nun mit Phishing Daten des Benutzer stehlen.

\subsubsection{Gegenmassnahme}
Der beste Schutz gegen diesen Angriff ist der Einsatz von DNSSEC, welcher mit Autentifizierung und Integrität arbeitet.
\subsection{Phishing}
\subsubsection{Bedrohung}
Beim Phishing versucht ein Angreifer durch gefälschte Websiten, SPAM Mails oder andere Methoden an Daten eines Internet-Benutzer zu gelangen. So kann ein Angreifer an Kreditkarteninformationen oder weitere Daten kommen und einen erheblichen finanziellen Schaden anrichten.
\subsubsection{Gegenmassnahme}
Leider gibt es gegen diese Attacke keine effektive Schutzmassnahme. Um sich möglichst gut gegen diese Attacke zu schützen, müssen die Benutzer geschult werden. Zudem kann ein SPAM Filter Mails von potentiellen Angreifern löschen oder markieren, sodass sich der Benutzer dem Risiko bewusst ist.

\subsection{MAC flooding}
\subsubsection{Bedrohung}
Ein Angreifer sendet viele ARP Antworten. Dabei setzt er immer eine andere MAC Adresse. Wenn die Index Tabelle des Switches voll ist, schaltet dieser in den Hub Modus um und sendet alle Packete jedem angeschlossenen Gerät. Nun kann der Anfreifer jegliche Kommunikation über diesen Switch mithören. 
\subsubsection{Gegenmassnahme}
Um sich gegen diese Attacke zu schützen, kann auf dem Switch definiert werden, dass er ausschalten soll, wenn die Index Tabelle voll ist. Dadurch ist zwar ein Unterbruch im Netz vorhanden, aber der Angreifer kann keine nicht mithören.\\
Eine noch besserer Schutz ist, wenn die Port Security auf dem Switch aktiviert und konfiguriert wird. Dadurch hat kein Angreifer die Möglichkeit die Index Tabelle des Switches zu füllen.

\subsection{ARP spoofing}
\subsubsection{Bedrohung}
Ein Agreifer sendet ARP Antworten mit den IP Adressen der Opfer und seiner eigenen MAC Adresse. Der Switch merkt sich nun dass die IP Adressen zur MAC Adresse des Angreifers gehören. Wenn nun ein Opfer ein Paket sendet, wird dieses vom Switch zum Angreifer weitergeleitet. Der Angreifer hat nun Einblick in die Daten, kann diese allenfalls verändern und leitet dieses schliesslich weiter zum effektiven Ziel, sodass niemand etwas davon mitbekommt.
\subsubsection{Gegenmassnahme}
Um sich gegen diese Attacke zu schützen, kann die Port Security auf dem Switch aktiviert werden, dadurch hat ein potentieller Anfreifer gar keine Möglichkeit sich ins interne Netz einzubinden.

\subsection{DHCP}
\subsubsection{Bedrohung}
Eine Person mit Zugriff auf ein Netzwerkkabel im internen Netz verbindet einen zusätzlichen, nicht autorisierten DHCP Server. Wenn der zusätzliche DHCP Sever schnellere Antwortzeiten hat als der offizielle DHCP Server, erhalten die Clients nun eine IP des nicht autorisierten DHCP Server, wodurch diese nicht mehr auf die interne Infrastruktur zugreiffen können.
\subsubsection{Gegenmassnahme}
Um dies zu verhindern, kann der Port 68 für DHCP Antworten blockiert werden (ausser vom offiziellen DHCP Server). Dadurch ist sichergestellt, dass kein zusätzlicher DHCP Server IP Adressen im interne Netz verteilen kann.

\subsection{Überblick}
Hier noch einmal ein Überblick der beschriebenen Bedrohungen inkl. Priorisierung und Markierung, gegen welche wir uns schützen:\\
!!!!Create Latex Table!!!!\\
Ranking	Eintrittswahrscheinlichkeit		Schweregrad		Bedrohung					Schutz umgesetzt\\
1		hoch					hoch			ICMP ‘smurf attack’: Denial of Service	ja\\
2		hoch					mittel			Viren / Würmer / Trojaner			nein\\
3		mittel					hoch			TCP DoS (Syn-Flooding)			ja\\
4		mittel					hoch			DNS Cache poisoning			nein\\
5		hoch					niedrig		Phishing					nein\\
6		niedrig				hoch			DHCP						ja\\
7		niedrig				mittel			MAC flooding					nein\\
8		niedrig				mittel			ARP spoofing				nein\\
\end{document}